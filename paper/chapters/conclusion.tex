\section{Conclusion} \label{chapter_conclusion}

HPC application design aspects have all been regarded by today's web browser technologies. Yet performance, completeness and support is different for each one.

Web-browser-based HPC applications have to be written in JavaScript. This work presented JavaScript as a flexible language for developers. Regarding performance, the type system and memory handling is essential. Benchmarks show performance deficits, as JavaScript is dynamically typed and wraps even CPU native data types.

Mozilla tries to tackle performance issues with asm.js. By annotating data, a web browser can interpret the data as certain CPU native data types. This allows the use of performance optimized routines internally instead of regular dynamic type resolution. Furthermore, Emscripten compiles C/C++ applications to asm.js. To circumvent Garbage Collection, a single array is used as a virtual heaps for all objects. Benchmarks show significant performance improvements to regular JavaScript. Firefox, Chrome and Internet Explorer detect asm.js. For performance critical code sections, asm.js generated by Emscripten is a solid solution for fast, portable applications.

Regarding intra-node processing, HTML5 Web Workers can be used for asynchronous computations using threads. Message passing has to be used for communication; shared memory is not possible. As structured cloning of complex objects is slow, transferable objects like arrays should be preferred for communication. Still, sender and receiver share no data. A sent message is only available to the receiver. Web Workers are supported by all web browsers, but Internet Explorer does unfortunately not support transferable objects communication.

Inter-node processing is restricted to TCP-based client-server architectures using HTML5 WebSockets. Google started WebRTC to allow arbitrary media transport between web browsers. Part of WebRTC is the DataChannel. It provides configurable reliability, delivery order and more. Additionally, by using a signalling server, peer-to-peer connection between web browsers are possible. Using WebRTC DataChannel, arbitrary communication patterns can be realized. Currently, DataChannels are only supported by Chrome and Firefox.

GPU computing is difficult with today's web browsers. Plugins for WebCL are available for Chrome and Firefox. Developers kowing OpenCL can immediately start writing kernels. But integration with a graphics pipeline of WebGL is difficult. While OpenGL provides Compute Shaders for GPU computing, they are not yet supported bei WebGL. If they once will be, graphics pipeline integration is implicit. But WebCL offers superior hardware eploitation for maximum performance and conforms to IEEE 754 floating point computations.

Finally, JavaScript performance aims to be near-native using asm.js. Intra-node and inter-node processing can be realized with HTML5 Web Workers and WebRTC DataChannel. Only GPU computing is hardly realizable with today's technologies. The author recommends using Chrome of Firefox when implementing an HPC application today.
