\section{Conclusion} \label{chapter_conclusion}

For performance critical code sections, asm.js generated by Emscripten provides fast, portable applications. While there are concurrent technologies, like Chrome's Native Client allowing compilation from C/C++, portability and profiting from ever improving JavaScript engines are strong arguments. But the toolchains to generate asm.js from C/C++ and integrate it in a web application need to be improved for broad adaption. This said, future shows if developers adapt to asm.js or rely on JavaScript engine optimizations.

Regarding intra-node processing, Internet Explorer urgently needs to adapt transferable objects for message passing.

WebRTC is not fully implemented by web browsers, but is expected to be in the future. Current implementations are open source and WebRTC is standardized. DataChannel is a reliable option for current and future HPC applications. Other features are going to allow the development of web applications hardly imaginable right now.

As for GPU computing, history might repeat itself. Mozilla wants to rely on OpenGL Compute Shaders of the graphics pipeline, just like early desktop GPU computing applications did. Still, standardization of WebCL is in progress. While adaption is slow right now, the author expects that once more complex games are developed for web browser, developers will be interested in WebCL support.

Regarding performance and feature completeness, the author recommends using Chrome or Firefox whem implementing web-browser-based HPC applications today.
