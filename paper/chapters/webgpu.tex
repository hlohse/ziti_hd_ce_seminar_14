\section{GPUs as Coprocessors} \label{chapter_webgpu}

GPUs have established themselves as efficient coprocessors for workloads requiring massively parallel SIMD computations. These workloads are part of many HPC applications. Yet, there is no native web browser support for doing arbitrary computations on GPUs. Possible candidates are discussed in the following subsections.


\subsection{WebCL}

WebCL is derived from OpenCL. The latter allows computations to be done on any OpenCL-compatible processor, like CPUs or GPUs. It is intended for heterogeneous computing purposes, but is commonly used for GPU computing where CUDA for Nvidia GPUs is no option.  WebCL is in the process of specification and aims to bring an OpenCL subset to JavaScript.

The advantages of using WebCL in a web browser are:

\begin{itemize}

\item \textit{Like OpenCL}. Developers confortable with OpenCL can immediately work with WebCL.

\item \textit{Hardware exposure}. WebCL does not hide hardware details behind abstractions. The developer has full access to the device's capabilites. Benefitial effects like memory access coalescing or avoiding bank conflicts can efficiently be implemented. This allows the best possible performance.

\item \textit{IEEE 754 float}. Floating point computations are conform to the IEEE 754 standard. This is especially important for scientific applications where a certain precision is mandatory. But other applications might also profit from non-irregular rounding effects, which are hard to debug or circumvent if arisen.

\end{itemize}

Disadvantegeous for adaption of WebCL is the way it operates. Like with OpenCL, the code to be run by the device, the so-called kernel, is provided as a string of characters. This gets passed to the OpenCL driver for just-in-time compilation and execution on the device. This requires a driver for every platform. Yet today, not even all available desktop GPUs have access to OpenCL drivers, like the HD 3000 integrated GPU in Intel's Sandy Bridge processors. This state if even worse for mobile platforms.

Additionally, WebCL would operate separately from the graphics pipeline. If an application utilized WebGL for rendering using the GPU, adding some WebCL execution efficiently is difficult. This separation and the fact that few developers are comfortable with GPU computing is the reason Mozilla does not plan to implement WebCL in Firefox. Other web browser developers have made no announcements. Currently, WebCL can only be used with plugins for Firefox or Chrome. \cite{webcl_ff}


\subsection{OpenGL Compute Shaders}

Since version 4.3 (3.1 for embedded systems, ES), OpenGL supports Compute Shaders. These are shader programs written in OpenGLs shader language GLSL allowing arbitrary computations. Every platform supporting the mentioned versions is capable of using Compute Shaders. Web browsers do not use OpenGL directly, but a derived subset called WebGL. Unfortunately, Compute Shaders are not part of WebGL version 1.0 and will not be part of version 2.0. It is not foreseeable when a web browser might utilize them.

Nevertheless, Compute Shaders might be adopted by developers once available. Advantegeous is that they are written in GLSL like graphics code, which developers may be familiar with. Integration into the graphics pipeline is implicit. Yet this may be the reason for problems, as e.g. graphics data types like textures have to be used for computations. The device's hardware is not fully exposed, capabilities are abstracted away. The resulting implementation can not force maximum optimizations. Additionally, floating point computations might not conform to IEEE 754. \cite{opengl_float}
